\chapter{Introduction}
\section{Modern Scientific Publishing}
The widespread adoption of the internet in the late 1990’s and 2000’s, brought  fundamental changes to the academic publishing landscape. The information revolution allowed publishers` costs to fall, and there was a mood shift in the academic sphere away from subscription based models, towards giving open and free access to some or all of journal article contents.
Simultaneously, institutions (such as university websites) began to post records of recent publications and other chemical information freely online. 
Publishers still protect the vast majority of journal article content and some metadata, as publication meta-data they is valuable and powerful. There is a well known saying in Data Science from

Tim O’Reilly, “The Guy with The Most Data Wins” CITE. As such, publishers are unwilling to grant free access their data for analysis by the public, preferring to perform in-house analysis. Article meta-data, such as authors, titles and abstracts may however be available, and it is this dataset which the project is focussed on. 
\section{Motivation}
By collecting metadata on papers found on the internet, a large, representative dataset of chemical academic writing language can be built up. Machine Learning techniques can then be applied to find novel connections between articles, research communities, authors, institutions and fields. Several publishers provide services that perform large scale analysis and search, such as SciFinder\textregistered and Web of Knowledge\textsuperscript{TM}. The techniques used and motivations behind the corporate bodies owning these services are not necessarily clear and thus there is much to be gained from independent, original analyses of the online publishing landscape.
\section{Aims}
The aims of the project are set out below:
\begin{itemize}
\item Collect large quantities of article meta-data from articles pertaining to chemistry as a general discipline
\begin{itemize}
\item Identify website that might contain useful chemical information
\item Write web-scraping programs that can scrape to identify and extract chemical information
\item Store information in human readable,computer readable, scalable and stable formats
\end{itemize}
\item Develop novel machine learning techniques to enable meta-data to be interpreted in new ways
\begin{itemize}
\itemsep0em 
\item Sanitise input data effectively
\item Devise models to interpret article titles and abstracts to attempt extract their chemical meaning
\item Quantitatively represent an article's content using its collected meta-data
\end{itemize}

\item Validate the model and provide evidence of their efficacy
\begin{itemize}
\item Develop visualisation techniques for interpretation of algorithm output
\item Devise tests of algorithm to ascertain whether it performs as hypothesised
\end{itemize}
\item Analyse datasets using the developed model to demonstrate new and useful information
\item Provide usable code for future analyses to be performed with
\end{itemize}
