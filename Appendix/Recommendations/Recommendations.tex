\label{chapt:RECOMMENDATIONS}
As alluded to in the text, there are several recommendations for further work. If carrying out further work on this project, it is recommended to contact the author for in-depth explanations. It is the author's belief that literature semantic analysis should be considered an important analytical chemical tool.
\subsection{Greater Dimensionality and Training Improvements}
Models should now be improved. Computing resources should be obtained to train higher dimensional vectors. The models should also be trained for more ($> 24$) epochs on more data ($> 460000$ documents) leading to more expressive models.
\subsection{Greater use of word vectors}
\label{sec:recomm_word_vectors}
This project focussed mainly on document vectors. However, word vectors may be very useful (see \S\ref{sec:Elemental_Analysis}). Methods for testing the quality of improved models should be develope, e.g  relationship-testing: e.g. \texttt{Fluorine} is to \texttt{Fluoride} as \texttt{Chlorine} is to \texttt{...} . Thousands of these relationships could be systematically built up to test model intuition\footnote{This would probably require larger, more descriptive training sets, e.g. textbook transcipts etc.} following the methodologies set out in the literature \cite{word2vec1}\cite{word2vec2}. Is it possible to predict chemical properties using semantic relationships in the literature? Vec(Compound A) + Vec(Compound B) + Vec(Lab Technique) may give vec(Product C). If so, it may be possible to highlight unexpected reactions. This could be coupled with the RInChI database to form new types of data-driven cheminformatics.
\newpage
\null
\newpage
\subsection{Time resolution in clustering}
Methods have been described for clustering documents. The cluster centres represent the cluster content effectively. By finding early papers in the cluster, is it possible to identify influential papers/authors?
By clustering on documents from particular years, is it possible to resolve a path for the evolving cluster centre vector? If so, it may be possible to extrapolate to \emph{predict} future research directions.
\subsection{Open Source Chemistry Vectors}
With the increase in open source papers, it should be possible to build up a vast dataset of chemical language for training, using the bodies of articles published on open source platforms, and even to use supplied supporting information. 
\subsection{Structure stemming}
Chemical names could be smartly preprocessed to classes of chemicals, for example by identifying a compound from its name and mapping to InChI key, then to a chemical class. This would allow better association of chemical fragments in training.
\subsection{Multiply labelled Documents}
In training Doc2Vec, by specifying documents with more than just unique identifiers allows more vectors to be trained. By identifying and labelling documents with a particular concept, e.g. `palladium-catalysed', and then training Doc2Vec, a `palladium-catalysed' concept vector can be trained. These concept vectors would be easily mine-able\footnote{e.g. documents close to the palladium-catalysed vector but do not contain the word palladium.}
\subsection{TSNE Maps}
There was not sufficient time to explore the clustering found by TSNE reductions. TSNE is a popular technique in the current machine learning literature, and should be investigated more thoroughly. K-Means clustering performed on the TSNE maps was briefly investigated before rapid progress was made by other techniques. There was evidence to suggest that TNSE K-Means clustering was potentially useful, but time did not permit investigation.