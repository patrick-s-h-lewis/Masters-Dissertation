\label{sec:collabtable}
In this section, the suggested collaboration table \ref{tab:topcollabs} is briefly explored to rationalise some of the suggested pairings.
Limitations should be explained. 
The recommended collaborations arise from a balance of two factors: how similar the pair are, and how often they have collaborated in the past.The evidence that members are collaborating is taken from \url{http://www.ch.cam.ac.uk/publications/authors}, finding articles that are in $\Delta7$ and considering authors to be collaborating if they co-author papers. This is not a particularly robust metric, as there were only about 700 co-authored paper found in this process, a small sample. This is why the main body of \S\ref{chapt:ANALYSIS} refers to `evidence of collaboration' rather than concretely stating that authors are collaborating. In order to build better metrics, citations and a wider body of co-authorship data would be required, which is beyond the scope of this project.

This said, the table is still useful as a guide to who these staff members should focus future collaboration with. It \emph{does not} assert that these authors are not already collaborating, only that they should treat the collaboration recommendation as a useful guide to who would be fruitful to work with, as their work is quite strongly related.

The top suggestion is for Dr. Archibald to work with Prof. Harris. They both work in the Centre of Atmospheric Science. There is one instance of collaboration on \url{http://www.ch.cam.ac.uk/publications/authors} but this article was not successfully collected to $\Delta7$. If this evidence was represented in the dataset, it is likely their recommendation score would drop. It is so high is because their work was considered very similar by the models, pushing their recommended collaboration score high.

David Wales and Daan Frenkel were second highest, and there is also evidence to suggest they have  collaborated in the past on \url{http://www.ch.cam.ac.uk/publications/authors}. Some of this evidence was represented in the collaboration matrix, but perhaps with more data, the association would have been stronger. Their research was considered very similar and this outweighed the evidence of collaboration to give a high score.

This trend (some evidence of collaboration, but strong similarity) is present in most of the top 20. 

It should be noted that date of publications was not a factor considered in the analysis. This goes some way to explaining why authors who were not simultaneously at the department for long periods have high recommended collaboration scores, as they would not have collaborated. For example, this is probably the case for Dr. Andrew Bond, who features several times in the table. As a recent staff member who specialises in crystallographic techniques, there has not been a great deal of time for collaboration and co-authorship. Many authors publish articles mentioning crystallography, and so Dr. Bond's work will have high similarity to several members in the department.
